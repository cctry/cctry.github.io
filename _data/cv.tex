%------------------------------------
% Dario Taraborelli
% Typesetting your academic CV in LaTeX
%
% URL: http://nitens.org/taraborelli/cvtex
% DISCLAIMER: This template is provided for free and without any guarantee 
% that it will correctly compile on your system if you have a non-standard  
% configuration.
% Some rights reserved: http://creativecommons.org/licenses/by-sa/3.0/
%------------------------------------

%!TEX TS-program = xelatex
%!TEX encoding = UTF-8 Unicode

\documentclass[12pt, a4paper]{article}
\usepackage{fontspec} 
\usepackage{siunitx}
\usepackage{enumitem}
% DOCUMENT LAYOUT
\usepackage{geometry} 
\geometry{a4paper, textwidth=6in, textheight=10in, marginparsep=10pt, left=40mm, marginparwidth=.8in}
\setlength\parindent{0in}

% FONTS
\usepackage[usenames,dvipsnames]{xcolor}
\usepackage{xunicode}
\usepackage{xltxtra}
\usepackage{multicol}
\usepackage{url}
\defaultfontfeatures{Mapping=tex-text}
%\setromanfont [Ligatures={Common}, Numbers={OldStyle}, Variant=01]{Linux Libertine O}
%\setmonofont[Scale=0.8]{Monaco}
%%% modified by Karol Kozioł for ShareLaTeX use
\setmainfont[
  Ligatures={Common}, Numbers={OldStyle}, Variant=01,
  BoldFont=LinLibertine_RB.otf,
  ItalicFont=LinLibertine_RI.otf,
  BoldItalicFont=LinLibertine_RBI.otf
]{LinLibertine_R.otf}
\setmonofont[Scale=0.8]{DejaVuSansMono.ttf}

% ---- CUSTOM COMMANDS
\chardef\&="E050
\newcommand{\html}[1]{\href{#1}{\scriptsize\textsc{[html]}}}
\newcommand{\pdf}[1]{\href{#1}{\scriptsize\textsc{[pdf]}}}
\newcommand{\doi}[1]{\href{#1}{\scriptsize\textsc{[doi]}}}
% ---- MARGIN YEARS
\usepackage{marginnote}
\newcommand{\amper{}}{\chardef\amper="E0BD }
\newcommand{\years}[1]{\marginnote{#1}}
\renewcommand*{\raggedleftmarginnote}{}
% \setlength{\marginparsep}{12pt}
\reversemarginpar

% HEADINGS
\usepackage{sectsty} 
\usepackage[normalem]{ulem} 
\sectionfont{\mdseries\upshape\Large}
\subsectionfont{\mdseries\scshape\normalsize} 
\subsubsectionfont{\mdseries\upshape\large} 

% PDF SETUP
% ---- FILL IN HERE THE DOC TITLE AND AUTHOR
\usepackage[%dvipdfm, 
bookmarks, colorlinks, breaklinks, 
% ---- FILL IN HERE THE TITLE AND AUTHOR
	pdftitle={Shiyang Chen - Curriculum Vitae},
	pdfauthor={Shiyang Chen}
]{hyperref}  
\hypersetup{linkcolor=blue,citecolor=blue,filecolor=black,urlcolor=MidnightBlue} 

% DOCUMENT
\begin{document}

{\Large {\bf Curriculum Vitae - Shiyang Chen}}

\rule{0.72\paperwidth}{0.8pt}


\vspace{0.5em}

Ph.D. Student\hspace{8.13cm} \texttt{978-856-9583}

Rutgers, The State University of New Jersey, \hspace{2.79cm} \href{mailto: shiyang.chen@rutgers.edu}{shiyang.chen@rutgers.edu}

New Brunswick, NJ 08854 \hspace{5.9cm} \href{https://github.com/cctry}{GitHub homepage}

% ACM student membership number: 1912133


\vspace*{-2mm}
\section*{EDUCATION}
\vspace*{-2mm}
% \years{2019 - \\Present} \textbf{Ph.D. Student} in \textbf{High-Performance Computing}\\
% Department of Electrical \& Computer Engineering\\
% \textit{Rutgers, The State University of New Jersey}\\
% Adviser: Hang Liu\\

\years{2023 - \\Present} \textbf{Ph.D. Student} in \textbf{High-Performance Computing}\\
Department of Electrical \& Computer Engineering\\
\textit{Rutgers, The State University of New Jersey}\\
Advisor: Hang Liu\\

\years{2020 - \\2022} \textbf{Master Student} in \textbf{High-Performance Computing}\\
Department of Electrical \& Computer Engineering\\
\textit{Stevens Institute of Technology}\\
Advisor: Hang Liu\\

% \years{2019 - \\2023} \textbf{Ph.D. Student} in \textbf{High-Performance Computing}\\
% Department of Electrical \& Computer Engineering\\
% \textit{Stevens Institute of Technology}\\
% Adviser: Hang Liu\\

\years{2015 - 2019} \textbf{B.E.} in \textbf{Electric Science and Technology}\\
School of Optical and Electronic Information \\
\textit{Huazhong University of Science \& Technology}\\

% \years{2012 - 2015} \textbf{High school}\\
% \textit{Shenzhen Hongling Middle School}

\vspace*{-2mm}
\section*{EXPERIENCES}
\vspace*{-2mm}

\years{2023 - \\Present}
		\textbf{Graduate Research Assistant}\\
		\textit{Rutgers, The State University of New Jersey}\\
		Adviser: Hang Liu\\
		
\years{Summer\\2020} \textbf{Research Intern}\\
\textit{Lawrence Livermore National Laboratory}\\
Mentor: Chunhua Liao \\
Project: \textbf{Monte Carlo sampling with emerging hardware.} 
\begin{itemize}
    \item Developed a GPU-accelerated Monte Carlo sampling framework, leveraging ray tracing cores and associated acceleration structures on NVIDIA GPUs.
    \item Reformulated sampling as a rendering problem, utilizing OptiX framework, Tensor Cores, and CUDA kernels to build and preprocess data structures.
    \item Achieved significant reduced turn-around time in quantum mechanism simulations.
\end{itemize}


\years{2022.6-12} \textbf{Applied Scientist Intern}\\
\textit{Amazon Web Service}\\
Mentor: Da Zheng \\
Project: \textbf{Online inference for temporal Graph Neural Network.} 
\begin{itemize}
    \item Built the system for low-latency, staleness-bounded temporal GNN inference on distributed, large-scale graphs.
    \item Designed a staleness-bounded request scheduling algorithm, enabling efficient real-time GNN inference with dynamic graph updates.
    \item Delivered the system, making it the first deployed solution supporting online temporal GNN inference in production.
    
\end{itemize}


\years{Summer\\2023} \textbf{Applied Scientist Intern}\\
\textit{Amazon Web Service}\\
Mentor: Xiang Song \\
Project: \textbf{Large-scale Graph Neural Network inference.} 
\begin{itemize}
    \item Developed an end-to-end large-scale GNN inference system, optimizing both graph construction and node embedding computation.
    \item Designed GPU kernels for efficient graph processing and GNN computations across distributed machines, scaling to up to 48 EC2 instances.
    \item The system was deployed in production and integrated into the open-sourced GraphStorm project. It reduces end-to-end processing time from days to under an hour.
\end{itemize}


\years{2023 - \\2024}
		\textbf{External Collabrator}\\
		\textit{Microsoft Deepspeed}\\
		Mentor: Shuaiwen Leon Song\\
Project: \textbf{Deepspeed AI for Science. Large Language Model inference.}
\begin{itemize}
\item Extended the DeepSpeed framework for AI-for-science workloads, optimizing models like OpenFold and AI2BMD with customized GPU kernels.
\item Developed GPU kernels and system optimizations for DeepSpeed-Inference, especially for model quantization.
\end{itemize}




\years{Summer\\2024}
		\textbf{Software Engineer Intern}\\
		\textit{ByteDance}\\
		Mentor: Wei Xu\\
Project: \textbf{Disaggregated LLM inference framework.} 
\begin{itemize}
\item Designed a PoC system based on vLLM to disaggregate prefill and decode stages in LLM inference.
\item Developed a high-throughput, low-overhead KV cache transfer network library, enabling efficient distributed inference.
\item Achieved near-linear latency reduction as the system scaled across multiple nodes.
\end{itemize}

% \vspace*{-2mm}


% \newpage

\section*{RESEARCH - PUBLICATIONS}
% \vspace*{-2.5mm}
% * indicate students advised by me
\vspace{-2mm}

\years{2021} \textbf{Shiyang Chen}, Shaoyi Huang, Santosh Pandey, Bingbing Li, Guang Gao, Long Zheng, Caiwen Ding and Hang Liu. "E.T.: Rethinking Transformer Models on GPUs". In \textit{Proceedings of the International Conference for High Performance Computing, Networking, Storage and Analysis (\textbf{SC})}. ACM, 2021.

\vspace*{2mm}

\years{2021} Hongwu Peng, \textbf{Shiyang Chen}, Zhepeng Wang, Junhuan Yang, Scott Weitze, Tong Geng, Ang Li, Jinbo Bi, Minghu Song, Weiwen Jiang, Hang Liu and Caiwen Ding. "Optimizing FPGA-based Accelerator Design for Large-Scale Molecular Similarity Search". In \textit{Proceedings of the International Conference On Computer Aided Design (\textbf{ICCAD})}. IEEE/ACM, 2021.

\vspace*{2mm}

\years{2021} Shaoyi Huang, \textbf{Shiyang Chen}, Hongwu Peng, Daniel Manu, Zhenglun Kong, Geng Yuan, Lei Yang, Shusen Wang, Hang Liu and Caiwen Ding. "HMC-TRAN: A Tensor-core Inspired Hierarchical Model Compression for Transformer-based DNNs on GPU". In \textit{Proceedings of Proceedings of the 2021 on Great Lakes Symposium on VLSI (\textbf{GLVLSI})}. ACM, 2021.

\vspace*{2mm}

\years{2022} Shaoyi Huang, Dongkuan Xu, Ian Yen, Yijue Wang, Sung-En Chang, Bingbing Li, \textbf{Shiyang Chen}, Mimi Xie, Sanguthevar Rajasekaran, Hang Liu and Caiwen Ding. "Sparse Progressive Distillation: Resolving Overfitting under Pretrain-and-Finetune Paradigm". In \textit{Proceedings of the 60th Annual Meeting of the Association for Computational Linguistics (\textbf{ACL})}. ACL, 2022.

\vspace*{2mm}

\years{2022} Hongwu Peng, Shaoyi Huang, \textbf{Shiyang Chen}, Bingbing Li, Tong Geng, Ang Li, Weiwen Jiang, Wujie Wen, Jinbo Bi, Hang Liu and Caiwen Ding. "A Length Adaptive Algorithm-Hardware Co-design of Transformer on FPGA Through Sparse Attention and Dynamic Pipelining". In \textit{2022 59th ACM/IEEE Design Automation Conference (\textbf{DAC})}. IEEE, 2022.

\vspace*{2mm}

\years{2023} Yifei Wang, \textbf{Shiyang Chen}, Guobin Chen, Ethan Shurberg, Hang Liu, Pengyu Hong. "Motif-Based Graph Representation Learning with Application to Chemical Molecules". In \textit{Informatics Vol. 10. No. 1}. MDPI, 2023.

\vspace*{2mm}

\years{2023} \textbf{Shiyang Chen}, Da Zheng, Caiwen Ding, Chengying Huan, Yuede Ji and Hang Liu. "\textsc{Tango}: rethinking quantization for graph neural network training on GPUs". In \textit{Proceedings of the International Conference for High Performance Computing, Networking, Storage and Analysis (\textbf{SC})}. ACM, 2023. 

\vspace*{2mm}

\years{2023} Wang Feng, \textbf{Shiyang Chen}, Hang Liu and Yuede Ji. "\textsc{PeeK}: A Prune-Centric Approach for $K$ Shortest Path Computation". In \textit{Proceedings of the International Conference for High Performance Computing, Networking, Storage and Analysis (\textbf{SC})}. ACM, 2023. 



\vspace*{2mm}

\years{2023} Shuaiwen Song, Bonnie Kruft, Minjia Zhang, Conglong Li, \textbf{Shiyang Chen}, Chengming Zhang, Masahiro Tanaka, Xiaoxia Wu, Mohammed AlQuraishi, Gustaf Ahdritz, Christina Floristean, Rick Stevens, Venkatram Vishwanath, Arvind Ramanathan, Sam Foreman, Kyle Hippe, Prasanna Balaprakash, Yuxiong He. "DeepSpeed4Science Initiative: Enabling Large-Scale Scientific Discovery through Sophisticated AI System Technologies". In \textit{\textbf{NeurIPS} 2023 AI for Science Workshop.}  

\vspace*{2mm}

\years{2023} Xiaoxia Wu, Haojun Xia, Stephen Youn, Zhen Zheng, \textbf{Shiyang Chen}, Arash Bakhtiari, Michael Wyatt, Reza Yazdani Aminabadi, Yuxiong He, Olatunji Ruwase, Leon Song, Zhewei Yao.
"ZeroQuant(4+2): Redefining LLMs Quantization with a New FP6-Centric Strategy for Diverse Generative Tasks". \textit{Arxiv Preprint}.

\vspace*{2mm}

\years{2024}
Chengying Huan, Yongchao Liu, Heng Zhang, Shuaiwen Song, Santosh Pandey, \textbf{Shiyang Chen}, Xiangfei Fang, Yue Jin, Baptiste Lepers, Yanjun Wu, Hang Liu. "TEA+: A Novel Temporal Graph Random Walk Engine With Hybrid Storage Architecture". In \textit{ACM Transactions on Architecture and Code Optimization (\textbf{TACO})}.

\vspace*{2mm}

\years{2024}

Haojun Xia, Zhen Zheng, Xiaoxia Wu, \textbf{Shiyang Chen}, Zhewei Yao, Stephen Youn, Arash Bakhtiari, Michael Wyatt, Donglin Zhuang, Zhongzhu Zhou , Olatunji Ruwase, Yuxiong He, Shuaiwen Leon Song. "FP6-LLM: Efficiently Serving Large Language Models Through FP6-Centric Algorithm-System Co-Design". In \textit{2024 USENIX Annual Technical Conference (\textbf{ATC})}, USENIX, 2024.

\vspace*{2mm}

\years{2024} Ahdritz, Gustaf, et al. "OpenFold: Retraining AlphaFold2 yields new insights into its learning mechanisms and capacity for generalization." \textit{Nature Methods}: 1-11.

\vspace*{2mm}

\years{2024} Hongwei Chen, \textbf{Shiyang Chen}, Joshua J. Turner, Adrian Feiguin. "Kernel fusion in atomistic spin dynamics simulations on Nvidia GPUs using tensor core." \textit{Journal of Computational Science} (2024): 102357.



\end{document}